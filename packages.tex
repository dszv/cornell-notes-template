%% PACKAGES
\usepackage[vmargin=2.54cm,
            hmargin=2.54cm,
            reversemarginpar,
            marginpar=4.5cm,
            marginparsep=-4.5cm]{geometry}
\usepackage{changepage}
\usepackage[fulladjust]{marginnote}
%
\usepackage[utf8]{inputenc}
\usepackage[spanish, english]{babel}
\usepackage{amssymb, amsthm, amsmath, amsfonts}
\usepackage{kpfonts}
%
\usepackage{microtype}
\usepackage{nicefrac}
%
\usepackage{derivative}
%
\usepackage{hyperref, url}
\usepackage[inline]{enumitem}
%
\usepackage{tcolorbox}
\usepackage{booktabs}
\usepackage{multirow, multicol}
%
\usepackage{graphicx}
\usepackage{float}
\usepackage[labelfont=bf]{caption}
%
\usepackage[numbers]{natbib}


%% FORMATTING
% Packages needed
\usepackage{lineno}
\usepackage{mathtools}
\usepackage{xpatch}

% Mathematical expressions formatting
\newtheoremstyle{mod1}              % name
    {8.0pt plus 2.0pt minus 4.0pt}  % Space above
    {8.0pt plus 2.0pt minus 4.0pt}  % Space below
    {\itshape}                      % Body font
    {}                              % Indent amount
    {\scshape}                      % Theorem head font
    {.}                             % Punctuation after theorem head
    {.5em}                          % Space after theorem head
    {}                              % Theorem head spec

\newtheoremstyle{mod2}              % name
    {8.0pt plus 2.0pt minus 4.0pt}  % Space above
    {8.0pt plus 2.0pt minus 4.0pt}  % Space below
    {}                              % Body font
    {}                              % Indent amount
    {\scshape}                      % Theorem head font
    {.}                             % Punctuation after theorem head
    {.5em}                          % Space after theorem head
    {}                              % Theorem head spec

% \theoremstyle{plain}
\theoremstyle{mod1}
\newtheorem{theorem}{Theorem}
\newtheorem*{theorem*}{Theorem} 
\newtheorem{corollary}{Corollary}
\newtheorem{lemma}{Lemma}

% \theoremstyle{definition}
\theoremstyle{mod2}
\newtheorem{definition}{Definition}
\newtheorem{proposition}{Proposition}
\newtheorem{example}{Example}
\newtheorem{exercise}{Exercise}

\theoremstyle{remark}
\newtheorem{remark}{Remark}

% Unindent theorem and proof
\makeatletter
    \let\sv@thm\@thm
    \def\@thm{\let\indent\relax\sv@thm}
\makeatother

\makeatletter
\renewenvironment{proof}[1][\proofname]{\par
  \pushQED{\qed}%
  \normalfont \topsep6\p@\@plus6\p@\relax
  \trivlist
  %\itemindent\normalparindent
  \item[\hskip\labelsep
        % \itshape
        \scshape
    #1\@addpunct{.}]\ignorespaces
}{%
  \popQED\endtrivlist\@endpefalse
}
\makeatother

% Mathtools package with temporary solution for unumbered equations
\mathtoolsset{showonlyrefs=true , showmanualtags=true}
\MHInternalSyntaxOn
\xpatchcmd{\MT_extended_tagform:n}{
  \@ifundefined{MT_r_\df@label}{}
}{%
 \@ifundefined{MT_r_\df@label}{\kern1sp}
}{}{\typeout{patch failed}}
\xpatchcmd{\MT_extended_tagform:n}{
\@ifundefined{MT_r_\df@label}{\global\MH_set_boolean_F:n {manual_tag}}
}{%
\@ifundefined{MT_r_\df@label}{\global\MH_set_boolean_F:n {manual_tag}\kern1sp}
}{}{\typeout{patch failed}}
\MHInternalSyntaxOff


%% NEW COMMANDS
% Summary and notes
\newtcolorbox{summ}{arc=0mm, colframe=black, colback=white, coltitle=black, fonttitle=\bfseries, title=Summary, attach title to upper={\ ---\ }}
\newenvironment{summary}
    {\vfill
    \begin{adjustwidth}{-5cm}{0cm}
    \begin{summ}
    }
    {
    \end{summ}
    \end{adjustwidth}
    \newpage
    }

\renewcommand*{\raggedleftmarginnote}{}
\newcommand{\note}[1]{\marginnote{#1}}

% Math
% Brackets
\DeclarePairedDelimiter{\qtyp}{(}{)}
\DeclarePairedDelimiter{\qtys}{[}{]}
\DeclarePairedDelimiter{\qtyb}{\{}{\}}

% Norms
\DeclarePairedDelimiter{\abs}{\lvert}{\rvert}%
\DeclarePairedDelimiter\norm{\lVert}{\rVert}%

\makeatletter
\let\oldqtyp\qtyp
\def\qtyp{\@ifstar{\oldqtyp}{\oldqtyp*}}
%
\let\oldqtys\qtys
\def\qtys{\@ifstar{\oldqtys}{\oldqtys*}}
%
\let\oldqtyb\qtyb
\def\qtyb{\@ifstar{\oldqtyb}{\oldqtyb*}}
%
\let\oldabs\abs
\def\abs{\@ifstar{\oldabs}{\oldabs*}}
%
\let\oldnorm\norm
\def\norm{\@ifstar{\oldnorm}{\oldnorm*}}
\makeatother

% Differentiation
\newcommand{\dd}[2][]{\mathrm{d}^{#1} {#2}}
\newcommand{\idd}[2][]{\, \mathrm{d}^{#1} {#2}\, }
\newcommand{\var}[2][]{\mathrm{\delta}^{#1} {#2}}
\newcommand{\Var}[1]{\mathrm{\Delta} {#1}}

% Unit vectors
\newcommand{\vbi }{\textbf{\textup{\i }}}
\newcommand{\vbj }{\textbf{\textup{\j }}}
